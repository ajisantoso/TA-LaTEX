{\centering
	\textbf{\large Abstrak}\\
	\vspace{0.5cm}
}

%\chapter*{Abstrak}
Kejadian gawat darurat adalah kondisi dimana respon dibutuhkan dari pasien, keluarga, atau siapapun yang dianggap mempunyai kewajiban untuk mengambil keputusan membawa pasien ke rumah sakit untuk tindakan medis segera mungkin. Dalam menentukan pemilihan Unit Gawat Darurat (UGD) rumah sakit tujuan harus mempertimbangkan beberapa kriteria. Pada penelitian ini dibangun sistem pakar untuk mendapatkan kesesuaian UGD. Kriteria yang menjadi pertimbangan pemilihan UGD adalah kondisi pasien, waktu tempuh, ketersediaan dokter jaga, ketersediaan ruang operasi, ada ruang inap dan ruang ICU rumah sakit. Sistem pakar yang diabangun atas model berbasis pengetahuan (knowledge base). Pengetahuan yang disimpan menjadi rule diolah dengan mesin inferensi menggunakan algoritma Rete. Implementasi algoritma Rete yaitu dengan membuat hubungan antar node pada graf asiklik yang membentuk suatu jaringan Rete. Hubungan antar node dirancang agar bisa menghilangkan redundansi proses komputasi dari satu siklus ke siklus selanjutnya. Melalui penggunaan algoritma Rete pada rule based system diperoleh kecepatan dan efisiensi komputasi dengan mengurangi usaha yang dilakukan untuk komputasi berulang dari conflict set setelah aturan (rule) dieksekusi.
  
\vspace{0.5 cm}
\begin{flushleft}
{\textbf{Kata Kunci:} sistem pakar, algoritma Rete, knowledge base, inference engine, unit gawat darurat}
\end{flushleft}